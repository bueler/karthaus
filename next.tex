% Copyright 2009--2012  Ed Bueler

\section{next steps}

\subsection{practicalities}

\begin{frame}[fragile]
\frametitle{what technical skills are needed for \\ numerical ice sheet modeling?}

you need:

\begin{itemize}
\item comfort in a technical computing environment, usually Unix, in which you need to know:
  \begin{itemize}\small
  \item[$\circ$] an editor,
  \item[$\circ$] a compiled language (Fortran or C),
  \item[$\circ$] a scripting/prototyping language (Matlab, Python, etc.), and
  \item[$\circ$] \emph{a version control system} (Subversion, git, etc.)
  \normalsize
  \end{itemize}
\item willingness to read math, numerical analysis, computer science books
\item \dots and willingness to ignor \emph{some} of the advice found there
\item know some tools for NetCDF files
\item get exposed to parallel computing
\item but, at the end of the day: \emph{physics}
\end{itemize}
\end{frame}


\begin{frame}
\frametitle{what technical skills are needed? 2}

\begin{itemize}
\item an important skill is to \emph{not} re-invent the wheel
\item \emph{never} re-invent the wheel for basic numerics:
  \begin{itemize}
  \item[$\circ$] \Matlab, Comsol, PETSc, libmesh, Elmer, Trilinos, Triangle, etc.~handle fundamental numerical linear algebra, mesh generation, finite element assembly and solve, etc.~tasks; don't even try to compete!
  \item[$\circ$] \dots except to write throw-away codes to help you \emph{understand} numerical ideas 
  \end{itemize}
\item sometimes there is no need to re-invent ice sheet modeling:
  \begin{itemize}
  \item[$\circ$] open source SIA-based comprehensive models: GLIMMER, SICOPOLIS
  \item[$\circ$] open source hybrid/higher-order/Stokes models: PISM, Elmer-ice, CISM
  \end{itemize}
\item glacier and ice sheet modeling is young!  there is much to do!
\end{itemize}
\end{frame}
