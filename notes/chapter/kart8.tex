%kart8.tex

\newcommand{\Matlab}{\textsc{Matlab}\xspace}
\newcommand{\bq}{\mathbf{q}}
\newcommand{\bu}{\mathbf{u}}
\newcommand{\bU}{\mathbf{U}}
\newcommand{\Div}{\nabla\cdot}
\newcommand{\devstress}{\tau}

\chapter{\lab{ch8}Numerical methods\index{numerical methods}}
{\it\Large{Ed Bueler}}
\vspace*{0.3in}
\section{\lab{sec8.1}Shallow flow}


\subsection{\lab{sec8.1.1} Slab-on-a-slope}
\subsection{\lab{sec8.1.2} Shallow ice approximation (SIA)}
  \subsection{\lab{sec8.1.3} Scope of this chapter}

 \section{\lab{sec8.2} Finite differences}
  
    \subsection{\lab{sec8.2.1} Heat equation}
     \subsection{\lab{sec8.2.2} Explicit schemes and instabilities}
     \subsection{\lab{sec8.2.3} Diffusion equations and SIA}
     \subsection{\lab{sec8.2.4} Adaptive time steps}
     \subsection{\lab{sec8.2.5} Exact solutions and verification}
     \subsection{\lab{sec8.2.6} Ice sheet models}
  
  \section{\lab{sec8.3} Solving the stress balance for ice shelves}
  
     \subsection{\lab{sec8.3.1} Shallow shelf approximation (SSA)}
     \subsection{\lab{sec8.3.2} Nonlinear iterations}
     \subsection{\lab{sec8.3.3} Linear algebraic systems}
     \subsection{\lab{sec8.3.4} Ice streams and the grounding line}

 \section{\lab{sec8.4} Improving flow models}

     \subsection{\lab{sec8.4.1} Mass continuity and kinematical equations}
     \subsection{\lab{sec8.4.2} Longitudinal averaging}
     \subsection{\lab{sec8.4.3} Hybrid models}
     \subsection{\lab{sec8.4.4} Less-shallow approximations, Stokes}
     \subsection{\lab{sec8.4.5} Temperature, melt, sliding}
     \subsection{\lab{sec8.4.6} Skills and tools}
     
 \section{\lab{sec8.5}Notes and references}

  

\bex

\item \lab{q8.1} Assume $f$ has continuous derivatives of all orders.  Show using Taylor's theorem:
  $$f'(x) = \frac{f(x+\Delta) - f(x-\Delta)}{2\Delta} + O(\Delta^2) \quad \text{and} \quad f''(x) = \frac{f(x+\Delta) - 2 f(x) + f(x-\Delta)}{\Delta^2} + O(\Delta^2).$$

\item \lab{q8.2} Sometimes we want finite difference approximations for derivatives in-between grid points.  Continuing exercise \textbf{1}, show $f'(x+(\Delta/2)) = (f(x+\Delta) - f(x))/\Delta + O(\Delta^2)$.

\item \lab{q8.3} Rewrite \texttt{heat.m} using \texttt{for} loops instead of colon notation.  (The only purpose here is to help understand colon notation.)

\item \lab{q8.4} The 1D explicit scheme \eqref{heat1Dfd} for the heat equation, namely $T_j^{n+1} = \mu T_{j+1}^n + (1 - 2 \mu) T_j^n + \mu T_{j-1}^n$, is averaging if stability criterion \eqref{stabcrit} holds.  But of course we must be stepping \emph{forward} in time.  Show that the scheme is not averaging for any values of $\Delta t < 0$ and $\Delta x > 0$.  Try running \texttt{heat.m} backward in time to see what happens.  In general \emph{there are no consistent stable numerical schemes for unstable PDE problems}.

\item \lab{q8.5} Show that when written as a formula for $T_j^{n+1}$, scheme \eqref{implicit1D} has only positive coefficients.  By looking into the literature as needed, explain why this shows it is unconditionally stable.

\item \lab{q8.6} \emph{This multi-part exercise concerns the numerical treatment of ``$\Div\left(D\,\grad\right)$''.} 
\renewcommand{\labelenumi}{(\alph{enumi})}
\begin{enumerate}
\item Show that if $D=D(x,y)$ and $u=u(x,y)$ then $\Div \left(D\, \grad u\right) = D \grad^2 u + \grad D \cdot \grad u$.
\item Write down the centered $O(\Delta t)+O(\Delta x^2)$ explicit finite difference method for the equation $u_t = D_0 u_{xx} + E_0 u_x$, assuming $D_0>0$ and $E_0$ are constant.  Solve the scheme for the unknown $u_j^{n+1}$. 
\item Stability for your method will occur if the right hand side from the last answer in part (b) has all positive coefficients.  If $|E_0| \gg D_0$, what does this say about $\Delta t$?
\item Why do we use the staggered grid for ``$\Div\left(D\,\grad\right)$,'' instead of expanding by the product rule as in part (a)?
\end{enumerate}

\item \lab{q8.7} Derive the Green's function of the 1D heat equation, namely $T = (4 \pi D t)^{-1/2}\, e^{-x^2/(4Dt)}$, which is a solution to $T_t=D T_{xx}$.  Start by supposing there is a solution of the form $T(t,x) = t^{-1/2} \phi(s)$ where $s=t^{-1/2}x$ is the similarity variable.  Thereby write down an ordinary differential equation for $\phi$, and solve it.

\item \lab{q8.8} In the text, and in code \texttt{verifysia.m}, we used Halfar's solution to verify our numerical scheme \texttt{siaflat.m}.  Create the analogous code \texttt{verifyheat.m} to use the Green's function of the 2D heat equation \eqref{heat2D}, namely $T(t,x,y) = (4 \pi D t)^{-1}\, e^{-(x^2+y^2)/(4Dt)}$, to verify \texttt{heatadapt.m}.  You can use the high quality approximation $e^{-A^2}\approx 0$ for $|A|>10$ to choose a rectangular domain in space for which you may use $T=0$ Dirichlet boundary conditions.

\item \lab{q8.9} Is P.~Halfar male or female, and what does the first initial ``P.'' stand for?

\item \lab{q8.10} Show that formula \eqref{halfar} solves \eqref{sia} in the case of flat bed ($h=H$) and zero climatic mass balance ($M=0$).  You will want to express divergence and gradient in polar coordinates.

\item \lab{q8.11} In the text it is claimed that any modification of \texttt{siaflat.m} will make the output of \texttt{verifysia.m} show non-convergence, e.g.~the reported average thickness error will not go to zero as the grid is refined.  By randomly altering lines of \texttt{siaflat.m}, or by other methods of your choice, evaluate this claim.

\item \lab{q8.12} Some output from \texttt{verifysia.m} has been suppressed in the text, including a map-plane view of the numerical ice thickness error.  Near the grounded margin of an ice sheet this error is much larger than elsewhere.  Why?  Would a higher-order model with moving margin on the same grid have significantly smaller thickness error, supposing we knew a relevant exact solution?

\item \lab{q8.13} Derive the surface kinematical equation \eqref{surfkine} from the mass continuity \eqref{masscont} and basal kinematical \eqref{basekine} equations.  Note that the Leibniz rule for differentiating integrals, mentioned in the text, is
  $$\frac{d}{dx}\left(\int_{g(x)}^{f(x)} h(x,y)\,dy\right) = f'(x) h(x,f(x)) - g'(x) h(x,g(x)) + \int_{g(x)}^{f(x)} h_x(x,y)\,dy.$$

\item \lab{q8.14} Let $C_s = A (\rho g (1-r)/4)^n$ and assume $x_g=0$ is the location of the grounding line.  Derive the two parts of the van der Veen exact ice shelf solution, namely
\begin{align*}
  u &= \left[ u_g^{n+1} + (C_s/M_0) \left((M_0 x + u_g H_g)^{n+1} - (u_g H_g)^{n+1}\right) \right]^{1/(n+1)}, \quad H = (M_0 x + u_g H_g) / u,
\end{align*}
Start from equation \eqref{ssafloat}, and use the fact $(H^2)_x = 2 H H_x$ to generate the first integral of \eqref{ssafloat}.  Also use boundary condition \eqref{calvingstress}.  On the other hand, note that the mass continuity equation $M_0=(uH)_x$ can be integrated to give the formula $uH = M_0 (x-x_g) + u_g H_g$ for the flux.  Find $u(x)$ first, then $H(x)$ is in terms of $u(x)$ as above.  These exact solutions are from \cite{vanderVeen83}.  They are used in code \texttt{exactshelf.m}.

\item \lab{q8.15} Modify the code \texttt{ssaflowline.m} to solve the ice stream SSA equation \eqref{ssa}.  For boundary conditions it would be reasonable to have fixed velocity at the upstream end, but keep the calving front boundary condition at the downstream end.  (I.e.~consider the case where the calving front is at the point where the ice stream reaches flotation.)  Build a test case.  Find an exact solution, if you can, to verify your code.

\eex

